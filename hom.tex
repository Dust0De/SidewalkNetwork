\documentclass[11pt]{article}

\begin{document}

\textbf{2 Introduction:}

The Culver Academies is a private boarding school with a long history,
established in 1894. Since then, the school has developed and expanded
gradually, with new demands and constructions. The road network that has
developed in these years is a complex transportation system that is not
necessarily developed to fit the needs of students today.

Most of the student body at the Culver Academies had probably been
shouted at by adults for crossing the grass rather than taking a detour
on the sidewalks. This is a result of frustration caused by the designs
of the pathways on campus, feeling the irritation of seeing the
destination drifting further away as we walked on the sidewalks. The
existing pathway system is not planned for the continued growth of
future transportation needs.

~~~~~~ To make everyday path planning more convenient for students, this
paper will explore the various methods to generate and optimize pathway
networks in the Culver Academies. Designing an efficient path network
has been a challenge due to the interdependent system consisting of
discrete nodes. The goal of the designer is to construct a pathway
system with minimum cost while satisfying traffic needs for members of
the Culver Academies.

~~~~~~ The problem of designing a network to meet certain constraints
while optimizing specific characteristics draws similarities to many
real-life applications in communication networks, water systems, and
many other problems that could be modeled by the interactions between
discrete individual nodes. This wide range of applications encouraged
the development of many approaches. In this paper, I will be
investigating the approaches of several agent-based and heuristic
algorithms and applying them to the setting of Culver Academies.

\textbf{3. Background Research}

\subsection{3.1} \subsection{Network Planning}

Designing a network is a complicated optimization problem typically
formulated as an Integer Linear Programming (ILP) problem with
considerations of performance, reliability, and cost. The discrete
nature of the problem makes it a challenging problem to solve, often
requiring human experts to set certain standards (Zhu et al, 2021). Many
of the design processes and algorithms in network planning are developed
to other fields of study, primarily in telecommunications.

In telecommunications, designing a network starts with designing the
topologies with regard to cost. This step outputs the physical network
and the locations of intermediate nodes. Then, the traffic will be
evaluated to formulate the logical links between nodes (Penttinen,
1999). In this study, there is no difference between the logical and
physical, but the process of topological design before optimizing based
on traffic can be applied to this problem.

\subsection{3.2 Graph Theory}

A road network is often represented by a graph, with destinations and
intersections as vertices, and roads as edges, denoted as \(G = (V,E)\).
\emph{V} denotes the finite collection of vertices
\(v_{1},v_{2},\ldots,v_{n}\  \in V\ \)and \emph{E} denotes the finite
collection of edges. An edge connecting two vertices \(v_{1},v_{2}\)
could be directed, represented by the ordered pair
\(\left( v_{1},v_{2} \right)\), or can be non-directed, where it can be
represented by a set \(\text{\{}v_{1}.v_{2}\text{\}}\). While a sidewalk
seems to be intuitively non-directed as one can walk in both ways, a
directed edge is also useful to represent direction of traffic at
certain times.

To better model the geographical distance and traffic capacity of the
road network, a weighted graph \((G,w)\) is used where \(G\) denotes the
graph representing the road network and \(w\) maps every element in
\(E\) to a real number. In other words, it simply assigns one or more
numbers to each edge.

The relationship between nodes can be represented by an adjacency
matrix. The adjacency matrix \(A\) for a graph \(G\) is an
\(n\  \times n\) matrix. For the vertices
\(v_{1},v_{2},\ldots,v_{n} \in V(G)\), if
\(\left( v_{i},v_{j} \right) \in E(G)\), meaning that there exist an
edge for \(v_{i}\) and \(v_{j}\), then \(A_{ij}\  = \ 1\), else
\(A_{ij} = 0\). For a weighted graph, \(A_{ij}\) could also be equal to
a weight value. (Griffin, 2023).

\textbf{4. Problem Description:}

\begin{figure}
\centering
\includegraphics[width=4.39137in,height=3.48844in]{media/image1.png}
\caption{Figure 4.1 An unconnected graph with nodes representing
buildings on the Culver campus.}
\end{figure}

Given the locations of 30 nodes which represents all buildings on campus
and the traffic requirements \(t_{ij}\) between them, the problem is the
determine which of the \(n(n - 1)\text{/}2\) possible edges to
construct, which means choosing from \(2^{n(n - 1)\text{/}2}\) possible
topologies (Corne et al., 2000).

The aim of the design should be to satisfy the traffic demands of
students and the cost demands presented to the school.

To better fit the problem into a graph theory model, several assumptions
are made regarding the graph representation of the path network:

\begin{enumerate}
\def\labelenumi{\arabic{enumi}.}
\item
  The path network is a 2-dimensional model, ignoring all differences in
  attitude.
\item
  All paths have the same traffic capacity.
\item
  Only buildings enclosed by Academy Road, State Road 10, and East Shore
  Drive will be included in this model.
\end{enumerate}

\textbf{5. Methodology:}

\subsection{5.1. Slime Mold Algorithm (SMA)}

The slime mold algorithm draws inspiration from \emph{Physarum
polycephalum}, a type of slime mold that can form biological networks as
a part of their foraging strategy to discover and exploit food sources
(Tero et al., 2010). These biological networks have been optimized
through many cycles of evolution, making it a reasonable approach to
design a transportation network.

\emph{Physarum} is an example of a plasmodium -\/- large, single-celled
organism containing multiple nuclei. Each individual organism explores
nearby regions but resolves into a discrete nutrient pathway network
connecting the food sources directly with each other, and sometimes
through additional intermediate points known as Steiner points to even
minimize the total length of network (Tero et al., 2010).
\emph{Physarum} approaches food by chemical compounds released in the
air. After locating the food source, it will first diffuse around the
environment to nearby food sources. After contact with a food source,
most of the individual plasmodium decays, keeping only the ones needed
to link the food source to other parts of the slime mold (Li et al.,
2020).

\begin{longtable}[]{@{}
  >{\raggedright\arraybackslash}p{(\linewidth - 2\tabcolsep) * \real{0.1621}}
  >{\raggedright\arraybackslash}p{(\linewidth - 2\tabcolsep) * \real{0.8379}}@{}}
\caption{Figure 5.1.1. A simplified flowchart showing different phases
of the slime mold algorithm, and their corresponding biological
processes (Li et al, 2020).}\tabularnewline
\toprule\noalign{}
\begin{minipage}[b]{\linewidth}\raggedright
\textbf{Symbol}
\end{minipage} & \begin{minipage}[b]{\linewidth}\raggedright
\textbf{Description}
\end{minipage} \\
\midrule\noalign{}
\endfirsthead
\toprule\noalign{}
\begin{minipage}[b]{\linewidth}\raggedright
\textbf{Symbol}
\end{minipage} & \begin{minipage}[b]{\linewidth}\raggedright
\textbf{Description}
\end{minipage} \\
\midrule\noalign{}
\endhead
\bottomrule\noalign{}
\endlastfoot
\(t\) & Current iteration \\
\(P_{\max}\) & Maximum Pheromone Density for a cell \\
\(\delta\) & Decay parameter \\
\(\left( C_{x},C_{y} \right)\) & Current position of a slime cell \\
\(\left( F_{x},F_{y} \right)\) & Position of target food \\
MOVE\_TH & Moving threshold for primary diffusion \\
DIFF\_TH & Diffusion threshold for secondary diffusion \\
\(P_{i,j}(t)\) & Pheromone level of cell on \((i,j)\ \)at iteration
\(t\) \\
\(d\) & Direction diffusion \\
\(D\) & Diffusion decay rate \\
\(P_{\text{food}}\) & Pheromone level when a slime cell reaches food \\
\end{longtable}

\textbf{Table 5.1.1} Notation used in the slime mold algorithm

The slime mold simulation on the Culver building complex will be
implemented using Python, using the model described in Zhang, 2022. The
simulated slime mold model, an \(N\  \times M\ \)lattice graph
\(G = (L,E)\), where \(L\) is the set of cells for the \(N\  \times M\)
lattice. In the simulation, the dimension of the lattice graph is
determined by the maximum horizontal and vertical distance between
buildings on Culver Academies.

An individual cell contains its location index \((i,j)\), its current
state, and its pheromone density. Each cell can have one of the
following states at a given iteration \(t\), represented as
\(\tau_{i,j}(t) \in \text{\{}0,1,2\text{\}}\).

\begin{itemize}
\item
  0 - Empty: Cells without any food or slime.
\item
  1 - Slime: Cells that are considered part of the slime mold, which
  would represent the paths on the pathway network.
\item
  2- Food: Cells containing food source, representing the nodes of the
  pathway network.
\end{itemize}

In the simulation, an initial slime is placed onto one arbitrary food
cell to start. It will then start searching for nearby unconnected food
sources. Each slime cell, although viewed together as a single organism,
is modeled as an individual agent, making its own decisions based on
local inputs. Each slime cell will first gather the shortest path to the
nearest food source, and will diffuse, by replicating into neighboring
cells, following these four features (Zhang, 2022):

\begin{enumerate}
\def\labelenumi{\arabic{enumi}.}
\item
  Diffusion into another cell decreases that cell's pheromone density.
\item
  The slime will prioritize diffusing to a neighboring cell closest to
  shortest path to the food source.
\item
  The slime will not diffuse if it is too far from a connected food
  source, or if the target cell's pheromone level is too low.
\item
  The slime's decision will also be affected by the state of the
  neighboring cell: food, empty, or slime.
\end{enumerate}

The diffusing mechanism can be categorized into primary and secondary
diffusion, where primary diffusion occurs along the shortest path
towards the food source, and secondary diffusion occurs when primary
diffusion is blocked.

Before diffusion happens, from all 8 possible directions (positive and
negative directions along the horizontal, vertical, or diagonal), one or
more direction indices \(d\) is selected if the direction points to a
food source. These indices are set as the primary direction for a slime
cell.

Diffusion occurs when the pheromone level in the slime cell exceeds the
moving threshold (\(P_{C_{x},C_{y}}(t) >\) MOVE\_TH), the slime cell
attempts to initiate primary diffusion to the neighboring cell denoted
by \((k,l)\). If the cell \((k,l)\) lies within the lattice grid \(G\)
and is empty, the diffusion will begin by setting \((k,l)\) to a slime
cell.

Secondary diffusion occurs when the slime cell cannot diffuse along the
primary direction. Secondary diffusion is random and occurs when the
distance if its distance to the nearest food is smaller than the
diffusion threshold (DIFF\_TH) (Cai et al, 2020). This process is much
slower to reach a food source in order to favor primary diffusion, which
uses the shortest distance.

For every iteration during diffusion, an amount of pheromone is lost in
a slime cell. The pheromone level of the starting cell is updated by
this exponential decay function, where \(D\) is the diffusion decay rate
and \(\delta\) is the decay parameter:

\[P_{C_{x},C_{y}}(t + 1) = P_{C_{x},C_{y}}(t) \times (1 - D \times \delta)\]

By diffusing into the neighboring cell \((k,l)\), the slime mold
deposits an amount of pheromone to the new slime cell since an empty
cell has no pheromone. The pheromone level is updated as follows to
ensure that the level of pheromone does not exceed the maximum level
\(P_{\max}\) (Zhang, 2022):

\[P_{k,l}(t + 1) = \min\left( P_{k,l}(t) + \frac{P_{C_{x},C_{y}}(t)}{D},P_{\max} \right)\]

This same function is used when a slime cell has less pheromone than its
neighbor to simulate the transfer of nutrients from a food source.

If any slime cell becomes adjacent to a food cell during an iteration,
its pheromone level is instantly set to a fixed value
\(P_{\text{food}}\). This represents a reward for finding food.

The processes described above will continue until it reaches the maximum
iteration set at the beginning of the simulation.

During diffusion, the slime mold system will decay simultaneously based
on its distance to a connected food source and pheromone density of the
cell it is on. The diffused slimes will deliver nutrients to other food
sources and increase pheromone levels in this process. If a slime cell
has a pheromone level below a certain point, it will decay into an empty
cell. This process ensures that only the most efficient nutrient
pathways are kept (Li, et al, 2020).

To model this, the activity of the slime cell is directly tied to its
pheromone level so that if no exterior pheromone from the food sources
is provided, it cannot move to a new location through diffusion and
becoming inactive. Therefore, to determine the best path generated by
the slime mold algorithm, I would only have to look at if any two food
sources is connected by a continuous track of slime cells with high
levels of pheromone.

Works Cited

Cai, Zhengying, et al. "A Node Selecting Approach for Traffic Network
Based on Artificial Slime Mold." \emph{IEEE Access}, vol. 8, 2020, pp.
8436-48, https://doi.org/10.1109/access.2020.2964002.

Corne, David, et al. \emph{Telecommunications Optimization : Heuristics
and Adaptive Techniques}. Wiley, 2000.

Griffin, Christopher. \emph{Applied Graph Theory : an Introduction with
Graph Optimization and Algebraic Graph Theory}. World Scientific, 2023.

Lee, Jason. \emph{NETWORK OPTIMIZATION USING LINEAR PROGRAMMING AND
REGRESSION}. 2016. U Oregon, Bachelor\textquotesingle s thesis.
\emph{University of Oregon},
scholarsbank.uoregon.edu/server/api/core/bitstreams/cff1b048-02b6-442a-9eae-3c2a7b170ead/content.
Accessed 1 Oct. 2024.

Li, Shimin, et al. "Slime Mould Algorithm: A New Method for Stochastic
Optimization." \emph{Future Generation Computer Systems}, vol. 111, Oct.
2020, pp. 300-23. \emph{ScienceDirect},
https://doi.org/10.1016/j.future.2020.03.055. Accessed 15 Oct. 2024.

Penttinen, Aleksi. Lecture. \emph{Helsinki University of Technology},
1999, www.netlab.tkk.fi/opetus/s38145/s99/lectures/lect10.pdf. Accessed
6 Oct. 2024.

Tero, Atsushi, et al. "Rules for Biologically Inspired Adaptive Network
Design." \emph{Science}, vol. 327, no. 5964, 2010, pp. 439--42.
\emph{JSTOR}, http://www.jstor.org/stable/40508592. Accessed 15 Oct.
2024.

Zhang, Wenxiao. "CITS4403 Project Report." U of Western Australia, 17
Oct. 2022. \emph{Github},
github.com/MoeBuTa/SlimeMould/blob/master/report/report.pdf. Accessed 23
Oct. 2024.

Zhu, Hang, et al. "Network planning with deep reinforcement learning."
\emph{SIGCOMM \textquotesingle21: Proceedings of the 2021 ACM SIGCOMM
2021 Conference}. \emph{ACM Digital Library},
dl.acm.org/doi/10.1145/3452296.3472902. Accessed 6 Oct. 2024.
\end{document}
